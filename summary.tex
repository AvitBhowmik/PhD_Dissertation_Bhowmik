\chapter{Summary}
\label{Summary}

In the new epoch of Anthropocene, global freshwater resources are experiencing extensive degradation from a multitude of stressors. Consequently, freshwater ecosystems are threatened by a considerable loss of biodiversity as well as substantial decrease in adequate and secured freshwater supply for human usage, not only on local scales, but also on regional to global scales. Large scale assessments of human and ecological impacts of freshwater degradation enable an integrated freshwater management as well as complement small scale approaches. Geographic information systems (GIS) and spatial statistics (SS) have shown considerable potential in ecological and ecotoxicological research to quantify stressor impacts on humans and ecological entitles, and disentangle the relationships between drivers and ecological entities on large scales through an integrated spatial-ecological approach. However, integration of GIS and SS with ecological and ecotoxicological models are scarce and hence the large scale spatial picture of the extent and magnitude of freshwater stressors as well as their human and ecological impacts is still opaque. This Ph.D. thesis contributes novel GIS and SS tools as well as adapts and advances available spatial models and integrates them with ecological models to enable large scale human and ecological impacts identification from freshwater degradation. The main aim was to identify and quantify the effects of stressors, i.e climate change and trace metals, on the freshwater assemblage structure and trait composition, and human health, respectively, on large scales, i.e. European and Asian freshwater networks.

The thesis starts with an introduction to the conceptual framework and objectives (chapter 1). It proceeds with outlining two novel open-source algorithms for quantification of the magnitude and effects of catchment scale stressors (chapter 2). The algorithms, i.e. jointly called ATRIC, automatically select an accumulation threshold for stream network extraction from digital elevation models (DEM) by assuring the highest concordance between DEM-derived and traditionally mapped stream networks. Moreover, they delineate catchments and upstream riparian corridors for given stream sampling points after snapping them to the DEM-derived stream network. ATRIC showed similar or better performance than the available comparable algorithms, and is capable of processing large scale datasets. It enables an integrated and transboundary management of freshwater resources by quantifying the magnitude of effects of catchment scale stressors. Spatially shifting temporal points (SSTP), outlined in chapter 3, estimates pooled within-time series (PTS) variograms by spatializing temporal data points and shifting them. Data were pooled by ensuring consistency of spatial structure and temporal stationarity within a time series, while pooling sufficient number of data points and increasing data density for a reliable variogram estimation. SSTP estimated PTS variograms showed higher precision than the available method. The method enables regional scale stressors quantification by filling spatial data gaps integrating temporal information in data scarce regions. In chapter 4, responses of the assumed climate-associated traits from six grouping features to 35 bioclimatic indices for five insect orders were compared, their potential for changing distribution pattern under future climate change was evaluated and the most influential climatic aspects were identified (chapter 4). Traits of temperature preference grouping feature and the insect order Ephemeroptera exhibited the strongest response to climate as well as the highest potential for changing distribution pattern, while seasonal radiation and moisture were the most influential climatic aspects that may drive a change in insect distribution pattern. The results contribute to the trait based freshwater monitoring and change prediction. In chapter 5, the concentrations of 10 trace metals in the drinking water sources were predicted and were compared with guideline values. In more than 53\% of the total area of Pakistan, inhabited by more than 74 million people, the drinking water was predicted to be at risk from multiple trace metal contamination. The results inform freshwater management by identifying potential hot spots. The last chapter (6) synthesizes the results and provides a comprehensive discussion on the four studies and on their relevance for freshwater resources conservation and management.
