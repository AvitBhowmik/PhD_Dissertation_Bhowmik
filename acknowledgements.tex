\chapter{Acknowledgements}
\label{Acknowledgements}

This thesis and the time I spent behind the studies comprising it, represent a big shift in my career and personal paradigms as I went through a transition of becoming a Scientist and precisely a spatial eco(toxico)logist. Therefore, I would like to start by appreciating myself for the courage of growing with this immense challenge and not paddling back.

My heartfelt gratitude, of course, goes to the person who continuously supported me to make this happen, my thesis supervisor, Professor Dr. Ralf B. Schäfer. Ralf did not only teach me Science but also self-criticism, and that there is no trade-off for the quality of scientific work. We had many igniting discussions, where we appreciated and criticized each other on a wide range of topics and eventually, I learned a lot. Thank you very much Ralf, for being a good supervisor, friend and continuously welcoming and encouraging me in the world of spatial ecology and ecotoxicology. 

I would also like to take this opportunity to thank my thesis examiner and external supervisor Professor Dr. Ralf Schulz, vice president, Universität Koblenz-Landau. He has highly encouraged me after I presented the results from my first study in the university, which has fueled my transition to a spatial ecotoxicologist. I would also like to thank him for the acceptance of the invitation to become my thesis examiner as well as for including me in a project on global scale pesticide effects that showed me the open door of opportunities.

I am grateful to my research collaborators, Dr. Markus Metz from Research and Innovation Centre -- Fondazione Edmund Mach, Italy and Syed Ali Musstjab Akber Shah Eqani from COMSATS Institute of Information Technology, Pakistan. Markus provided continuous guidance and review while developing my first open-source computer algorithms and valuable comments on the article. Ali conceived the study on the nationwide risk assessment from trace metal contamination and provided data support and knowledge on Pakistan.

Universität Koblenz-Landau has awarded the scholarship to carry out my Ph.D. and also provided financial supports for conference attendance and open-access publications, and hence, I would like to thank the University authority.

I am strongly committed to promote ``open-science'', and hence only used open-source software for all analyses and made my tools and sample data freely available for ensuring reproducibility. R, GRASS GIS, \LaTeX, Libre-office, GitHub and PANAGEA were my means to stick to my commitment. Hence, a big hand goes to the developers and maintainers of these software and repositories. Especially, I would like to thank the big R community for contributing the useful ``spatial packages'' and answering a lot of questions crucial for my analyses.

My sincere thanks go to my colleagues at the institute for helping me towards the course of Ph.D. and explaining and clarifying the ecological jargons and suggesting the literature to look at. We had good times at birthday lunches, evening beers and Chirstmas markets. Thanks John, Bonny and Edi for your comments on my papers, and looking at the codes for analyses. Thanks Verena for proof-reading this thesis. We also had intriguing discussions on versatile topics that enlightened me throughout.

I would like to thank my parents, Shubhash and Pompy (both statisticians and that was my first inspiration to study statistics), and my brother, Bijoy: {\bengalifont অনেক অনেক ধন্যবাদ তোমাদের আশির্বাদ, সহযোগিতা আর প্রেরণার জন্য।}. It is their unconditional blessings, support and inspiration what made me who I am and brought me where I am today.

I cannot thank enough my fiancée, Caroline, for being my fixed point in life despite the ups and downs, unconditionally loving, supporting and encouraging me, proof reading: starting from my Ph.D. application to this dissertation, and familiarizing and adjusting me to German systems. This accomplishment was impossible without you and it will always be an inspiration for our future journeys together, no matter what!